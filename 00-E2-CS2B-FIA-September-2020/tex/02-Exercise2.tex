\documentclass[a4paper,12pt]{article}
%%%%%%%%%%%%%%%%%%%%%%%%%%%%%%%%%%%%%%%%%%%%%%%%%%%%%%%%%%%%%%%%%%%%%%%%%%%%%%%%%%%%%%%%%%%%%%%%%%%%%%%%%%%%%%%%%%%%%%%%%%%%%%%%%%%%%%%%%%%%%%%%%%%%%%%%%%%%%%%%%%%%%%%%%%%%%%%%%%%%%%%%%%%%%%%%%%%%%%%%%%%%%%%%%%%%%%%%%%%%%%%%%%%%%%%%%%%%%%%%%%%%%%%%%%%%
\usepackage{eurosym}
\usepackage{vmargin}
\usepackage{amsmath}
\usepackage{graphics}
\usepackage{epsfig}
\usepackage{enumerate}
\usepackage{multicol}
\usepackage{subfigure}
\usepackage{fancyhdr}
\usepackage{listings}
\usepackage{framed}
\usepackage{graphicx}
\usepackage{amsmath}
\usepackage{chngpage}
%\usepackage{bigints}

\usepackage{vmargin}
% left top textwidth textheight headheight
% headsep footheight footskip
\setmargins{2.0cm}{2.5cm}{16 cm}{22cm}{0.5cm}{0cm}{1cm}{1cm}
\renewcommand{\baselinestretch}{1.3}

\setcounter{MaxMatrixCols}{10}

\begin{document}
CS2B S2020–3 2Before answering this question, generate the vector, X, in R using the following code:set.seed(1027); X = rexp(n=1000, rate=0.01)The vector X represents the gross claim sizes of 1,000 claims. The payments are to be split between an insurance company and its reinsurer under an Excess of Loss reinsurance arrangement with a retention level M = 400. (i)Determine the proportion of the claims that are fully covered by the insurer. [2](ii)Generate an additional vector, Y, which is of the same length as X, such that Yrepresents the amounts to be paid by the insurer for each component of X.  [1] (iii)Generate an additional vector, ��, which is of the same length as X, such that ��represents the amounts to be paid by the reinsurer for each component of X. [1]An actuary assumes that the underlying gross claims distribution follows an exponential distribution of some unknown rate λ. The actuary needs to estimate λ using only the claim amounts recorded in vector Y.  (iv)Construct R code that calculates the log-likelihood, as a function of the parameter λ, given the claim amounts data in vector Y. [10] (v)Determine the value of λ at which the log-likelihood function reaches its maximum. [6] [Total 20] 
