\documentclass[a4paper,12pt]{article}
%%%%%%%%%%%%%%%%%%%%%%%%%%%%%%%%%%%%%%%%%%%%%%%%%%%%%%%%%%%%%%%%%%%%%%%%%%%%%%%%%%%%%%%%%%%%%%%%%%%%%%%%%%%%%%%%%%%%%%%%%%%%%%%%%%%%%%%%%%%%%%%%%%%%%%%%%%%%%%%%%%%%%%%%%%%%%%%%%%%%%%%%%%%%%%%%%%%%%%%%%%%%%%%%%%%%%%%%%%%%%%%%%%%%%%%%%%%%%%%%%%%%%%%%%%%%
\usepackage{eurosym}
\usepackage{vmargin}
\usepackage{amsmath}
\usepackage{graphics}
\usepackage{epsfig}
\usepackage{enumerate}
\usepackage{multicol}
\usepackage{subfigure}
\usepackage{fancyhdr}
\usepackage{listings}
\usepackage{framed}
\usepackage{graphicx}
\usepackage{amsmath}
\usepackage{chngpage}
%\usepackage{bigints}

\usepackage{vmargin}
% left top textwidth textheight headheight
% headsep footheight footskip
\setmargins{2.0cm}{2.5cm}{16 cm}{22cm}{0.5cm}{0cm}{1cm}{1cm}
\renewcommand{\baselinestretch}{1.3}

\setcounter{MaxMatrixCols}{10}

\begin{document}

Consider a set of data generated by tossing a fair coin, letting x t = 1 when a head is obtained
and x t = −1 when a tail is obtained. Then, construct y t as
$$y t = 2 + 1.5x t − 0.5x t−1$$
Simulate data for three cases, one with a small sample size (n = 10), another one with a
moderate sample size (n = 100) and lastly with large sample size (n = 1000).


## Exercise 1
i) Plot the series for all three cases.



```R

set.seed(101010)
x1 = 2*rbinom(11, 1, .5) - 1 # simulated sequence of coin tosses
x2 = 2*rbinom(101, 1, .5) - 1
x3 = 2*rbinom(1001, 1, .5) - 1
```


```R
y1 = 2 + filter(x1, sides=1, filter=c(1.5,-.5))[-1]
y2 = 2 + filter(x2, sides=1, filter=c(1.5,-.5))[-1]
y3 = 2 + filter(x3, sides=1, filter=c(1.5,-.5))[-1]
plot.ts(y1, type='s')
plot.ts(y2, type='s')
plot.ts(y3, type='s')





```


![png](output_2_0.png)



![png](output_2_1.png)



![png](output_2_2.png)


## Exercise 2 

Find the mean and variance and comment on results.




```R



c(mean(y1), mean(y2),mean(y3)) # the sample mean
c(var(y1), var(y2),var(y3)) # the variance


> c(mean(y1), mean(y2),mean(y3)) # the sample mean
 1.900 1.970 1.975
> c(var(y1), var(y2),var(y3)) # the variance
 3.433333 2.615253 2.582958
As the sample size increases, volatility reduces.

```
