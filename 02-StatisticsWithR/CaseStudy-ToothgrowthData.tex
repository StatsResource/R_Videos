
    




    
\documentclass[11pt]{article}

    
    \usepackage[breakable]{tcolorbox}
    \tcbset{nobeforeafter} % prevents tcolorboxes being placing in paragraphs
    \usepackage{float}
    \floatplacement{figure}{H} % forces figures to be placed at the correct location
    
    \usepackage[T1]{fontenc}
    % Nicer default font (+ math font) than Computer Modern for most use cases
    \usepackage{mathpazo}

    % Basic figure setup, for now with no caption control since it's done
    % automatically by Pandoc (which extracts ![](path) syntax from Markdown).
    \usepackage{graphicx}
    % We will generate all images so they have a width \maxwidth. This means
    % that they will get their normal width if they fit onto the page, but
    % are scaled down if they would overflow the margins.
    \makeatletter
    \def\maxwidth{\ifdim\Gin@nat@width>\linewidth\linewidth
    \else\Gin@nat@width\fi}
    \makeatother
    \let\Oldincludegraphics\includegraphics
    % Set max figure width to be 80% of text width, for now hardcoded.
    \renewcommand{\includegraphics}[1]{\Oldincludegraphics[width=.8\maxwidth]{#1}}
    % Ensure that by default, figures have no caption (until we provide a
    % proper Figure object with a Caption API and a way to capture that
    % in the conversion process - todo).
    \usepackage{caption}
    \DeclareCaptionLabelFormat{nolabel}{}
    \captionsetup{labelformat=nolabel}

    \usepackage{adjustbox} % Used to constrain images to a maximum size 
    \usepackage{xcolor} % Allow colors to be defined
    \usepackage{enumerate} % Needed for markdown enumerations to work
    \usepackage{geometry} % Used to adjust the document margins
    \usepackage{amsmath} % Equations
    \usepackage{amssymb} % Equations
    \usepackage{textcomp} % defines textquotesingle
    % Hack from http://tex.stackexchange.com/a/47451/13684:
    \AtBeginDocument{%
        \def\PYZsq{\textquotesingle}% Upright quotes in Pygmentized code
    }
    \usepackage{upquote} % Upright quotes for verbatim code
    \usepackage{eurosym} % defines \euro
    \usepackage[mathletters]{ucs} % Extended unicode (utf-8) support
    \usepackage[utf8x]{inputenc} % Allow utf-8 characters in the tex document
    \usepackage{fancyvrb} % verbatim replacement that allows latex
    \usepackage{grffile} % extends the file name processing of package graphics 
                         % to support a larger range 
    % The hyperref package gives us a pdf with properly built
    % internal navigation ('pdf bookmarks' for the table of contents,
    % internal cross-reference links, web links for URLs, etc.)
    \usepackage{hyperref}
    \usepackage{longtable} % longtable support required by pandoc >1.10
    \usepackage{booktabs}  % table support for pandoc > 1.12.2
    \usepackage[inline]{enumitem} % IRkernel/repr support (it uses the enumerate* environment)
    \usepackage[normalem]{ulem} % ulem is needed to support strikethroughs (\sout)
                                % normalem makes italics be italics, not underlines
    \usepackage{mathrsfs}
    

    
    % Colors for the hyperref package
    \definecolor{urlcolor}{rgb}{0,.145,.698}
    \definecolor{linkcolor}{rgb}{.71,0.21,0.01}
    \definecolor{citecolor}{rgb}{.12,.54,.11}

    % ANSI colors
    \definecolor{ansi-black}{HTML}{3E424D}
    \definecolor{ansi-black-intense}{HTML}{282C36}
    \definecolor{ansi-red}{HTML}{E75C58}
    \definecolor{ansi-red-intense}{HTML}{B22B31}
    \definecolor{ansi-green}{HTML}{00A250}
    \definecolor{ansi-green-intense}{HTML}{007427}
    \definecolor{ansi-yellow}{HTML}{DDB62B}
    \definecolor{ansi-yellow-intense}{HTML}{B27D12}
    \definecolor{ansi-blue}{HTML}{208FFB}
    \definecolor{ansi-blue-intense}{HTML}{0065CA}
    \definecolor{ansi-magenta}{HTML}{D160C4}
    \definecolor{ansi-magenta-intense}{HTML}{A03196}
    \definecolor{ansi-cyan}{HTML}{60C6C8}
    \definecolor{ansi-cyan-intense}{HTML}{258F8F}
    \definecolor{ansi-white}{HTML}{C5C1B4}
    \definecolor{ansi-white-intense}{HTML}{A1A6B2}
    \definecolor{ansi-default-inverse-fg}{HTML}{FFFFFF}
    \definecolor{ansi-default-inverse-bg}{HTML}{000000}

    % commands and environments needed by pandoc snippets
    % extracted from the output of `pandoc -s`
    \providecommand{\tightlist}{%
      \setlength{\itemsep}{0pt}\setlength{\parskip}{0pt}}
    \DefineVerbatimEnvironment{Highlighting}{Verbatim}{commandchars=\\\{\}}
    % Add ',fontsize=\small' for more characters per line
    \newenvironment{Shaded}{}{}
    \newcommand{\KeywordTok}[1]{\textcolor[rgb]{0.00,0.44,0.13}{\textbf{{#1}}}}
    \newcommand{\DataTypeTok}[1]{\textcolor[rgb]{0.56,0.13,0.00}{{#1}}}
    \newcommand{\DecValTok}[1]{\textcolor[rgb]{0.25,0.63,0.44}{{#1}}}
    \newcommand{\BaseNTok}[1]{\textcolor[rgb]{0.25,0.63,0.44}{{#1}}}
    \newcommand{\FloatTok}[1]{\textcolor[rgb]{0.25,0.63,0.44}{{#1}}}
    \newcommand{\CharTok}[1]{\textcolor[rgb]{0.25,0.44,0.63}{{#1}}}
    \newcommand{\StringTok}[1]{\textcolor[rgb]{0.25,0.44,0.63}{{#1}}}
    \newcommand{\CommentTok}[1]{\textcolor[rgb]{0.38,0.63,0.69}{\textit{{#1}}}}
    \newcommand{\OtherTok}[1]{\textcolor[rgb]{0.00,0.44,0.13}{{#1}}}
    \newcommand{\AlertTok}[1]{\textcolor[rgb]{1.00,0.00,0.00}{\textbf{{#1}}}}
    \newcommand{\FunctionTok}[1]{\textcolor[rgb]{0.02,0.16,0.49}{{#1}}}
    \newcommand{\RegionMarkerTok}[1]{{#1}}
    \newcommand{\ErrorTok}[1]{\textcolor[rgb]{1.00,0.00,0.00}{\textbf{{#1}}}}
    \newcommand{\NormalTok}[1]{{#1}}
    
    % Additional commands for more recent versions of Pandoc
    \newcommand{\ConstantTok}[1]{\textcolor[rgb]{0.53,0.00,0.00}{{#1}}}
    \newcommand{\SpecialCharTok}[1]{\textcolor[rgb]{0.25,0.44,0.63}{{#1}}}
    \newcommand{\VerbatimStringTok}[1]{\textcolor[rgb]{0.25,0.44,0.63}{{#1}}}
    \newcommand{\SpecialStringTok}[1]{\textcolor[rgb]{0.73,0.40,0.53}{{#1}}}
    \newcommand{\ImportTok}[1]{{#1}}
    \newcommand{\DocumentationTok}[1]{\textcolor[rgb]{0.73,0.13,0.13}{\textit{{#1}}}}
    \newcommand{\AnnotationTok}[1]{\textcolor[rgb]{0.38,0.63,0.69}{\textbf{\textit{{#1}}}}}
    \newcommand{\CommentVarTok}[1]{\textcolor[rgb]{0.38,0.63,0.69}{\textbf{\textit{{#1}}}}}
    \newcommand{\VariableTok}[1]{\textcolor[rgb]{0.10,0.09,0.49}{{#1}}}
    \newcommand{\ControlFlowTok}[1]{\textcolor[rgb]{0.00,0.44,0.13}{\textbf{{#1}}}}
    \newcommand{\OperatorTok}[1]{\textcolor[rgb]{0.40,0.40,0.40}{{#1}}}
    \newcommand{\BuiltInTok}[1]{{#1}}
    \newcommand{\ExtensionTok}[1]{{#1}}
    \newcommand{\PreprocessorTok}[1]{\textcolor[rgb]{0.74,0.48,0.00}{{#1}}}
    \newcommand{\AttributeTok}[1]{\textcolor[rgb]{0.49,0.56,0.16}{{#1}}}
    \newcommand{\InformationTok}[1]{\textcolor[rgb]{0.38,0.63,0.69}{\textbf{\textit{{#1}}}}}
    \newcommand{\WarningTok}[1]{\textcolor[rgb]{0.38,0.63,0.69}{\textbf{\textit{{#1}}}}}
    
    
    % Define a nice break command that doesn't care if a line doesn't already
    % exist.
    \def\br{\hspace*{\fill} \\* }
    % Math Jax compatibility definitions
    \def\gt{>}
    \def\lt{<}
    \let\Oldtex\TeX
    \let\Oldlatex\LaTeX
    \renewcommand{\TeX}{\textrm{\Oldtex}}
    \renewcommand{\LaTeX}{\textrm{\Oldlatex}}
    % Document parameters
    % Document title
    \title{CaseStudy-ToothgrowthData}
    
    
    
    
    
% Pygments definitions
\makeatletter
\def\PY@reset{\let\PY@it=\relax \let\PY@bf=\relax%
    \let\PY@ul=\relax \let\PY@tc=\relax%
    \let\PY@bc=\relax \let\PY@ff=\relax}
\def\PY@tok#1{\csname PY@tok@#1\endcsname}
\def\PY@toks#1+{\ifx\relax#1\empty\else%
    \PY@tok{#1}\expandafter\PY@toks\fi}
\def\PY@do#1{\PY@bc{\PY@tc{\PY@ul{%
    \PY@it{\PY@bf{\PY@ff{#1}}}}}}}
\def\PY#1#2{\PY@reset\PY@toks#1+\relax+\PY@do{#2}}

\expandafter\def\csname PY@tok@gd\endcsname{\def\PY@tc##1{\textcolor[rgb]{0.63,0.00,0.00}{##1}}}
\expandafter\def\csname PY@tok@gu\endcsname{\let\PY@bf=\textbf\def\PY@tc##1{\textcolor[rgb]{0.50,0.00,0.50}{##1}}}
\expandafter\def\csname PY@tok@gt\endcsname{\def\PY@tc##1{\textcolor[rgb]{0.00,0.27,0.87}{##1}}}
\expandafter\def\csname PY@tok@gs\endcsname{\let\PY@bf=\textbf}
\expandafter\def\csname PY@tok@gr\endcsname{\def\PY@tc##1{\textcolor[rgb]{1.00,0.00,0.00}{##1}}}
\expandafter\def\csname PY@tok@cm\endcsname{\let\PY@it=\textit\def\PY@tc##1{\textcolor[rgb]{0.25,0.50,0.50}{##1}}}
\expandafter\def\csname PY@tok@vg\endcsname{\def\PY@tc##1{\textcolor[rgb]{0.10,0.09,0.49}{##1}}}
\expandafter\def\csname PY@tok@vi\endcsname{\def\PY@tc##1{\textcolor[rgb]{0.10,0.09,0.49}{##1}}}
\expandafter\def\csname PY@tok@vm\endcsname{\def\PY@tc##1{\textcolor[rgb]{0.10,0.09,0.49}{##1}}}
\expandafter\def\csname PY@tok@mh\endcsname{\def\PY@tc##1{\textcolor[rgb]{0.40,0.40,0.40}{##1}}}
\expandafter\def\csname PY@tok@cs\endcsname{\let\PY@it=\textit\def\PY@tc##1{\textcolor[rgb]{0.25,0.50,0.50}{##1}}}
\expandafter\def\csname PY@tok@ge\endcsname{\let\PY@it=\textit}
\expandafter\def\csname PY@tok@vc\endcsname{\def\PY@tc##1{\textcolor[rgb]{0.10,0.09,0.49}{##1}}}
\expandafter\def\csname PY@tok@il\endcsname{\def\PY@tc##1{\textcolor[rgb]{0.40,0.40,0.40}{##1}}}
\expandafter\def\csname PY@tok@go\endcsname{\def\PY@tc##1{\textcolor[rgb]{0.53,0.53,0.53}{##1}}}
\expandafter\def\csname PY@tok@cp\endcsname{\def\PY@tc##1{\textcolor[rgb]{0.74,0.48,0.00}{##1}}}
\expandafter\def\csname PY@tok@gi\endcsname{\def\PY@tc##1{\textcolor[rgb]{0.00,0.63,0.00}{##1}}}
\expandafter\def\csname PY@tok@gh\endcsname{\let\PY@bf=\textbf\def\PY@tc##1{\textcolor[rgb]{0.00,0.00,0.50}{##1}}}
\expandafter\def\csname PY@tok@ni\endcsname{\let\PY@bf=\textbf\def\PY@tc##1{\textcolor[rgb]{0.60,0.60,0.60}{##1}}}
\expandafter\def\csname PY@tok@nl\endcsname{\def\PY@tc##1{\textcolor[rgb]{0.63,0.63,0.00}{##1}}}
\expandafter\def\csname PY@tok@nn\endcsname{\let\PY@bf=\textbf\def\PY@tc##1{\textcolor[rgb]{0.00,0.00,1.00}{##1}}}
\expandafter\def\csname PY@tok@no\endcsname{\def\PY@tc##1{\textcolor[rgb]{0.53,0.00,0.00}{##1}}}
\expandafter\def\csname PY@tok@na\endcsname{\def\PY@tc##1{\textcolor[rgb]{0.49,0.56,0.16}{##1}}}
\expandafter\def\csname PY@tok@nb\endcsname{\def\PY@tc##1{\textcolor[rgb]{0.00,0.50,0.00}{##1}}}
\expandafter\def\csname PY@tok@nc\endcsname{\let\PY@bf=\textbf\def\PY@tc##1{\textcolor[rgb]{0.00,0.00,1.00}{##1}}}
\expandafter\def\csname PY@tok@nd\endcsname{\def\PY@tc##1{\textcolor[rgb]{0.67,0.13,1.00}{##1}}}
\expandafter\def\csname PY@tok@ne\endcsname{\let\PY@bf=\textbf\def\PY@tc##1{\textcolor[rgb]{0.82,0.25,0.23}{##1}}}
\expandafter\def\csname PY@tok@nf\endcsname{\def\PY@tc##1{\textcolor[rgb]{0.00,0.00,1.00}{##1}}}
\expandafter\def\csname PY@tok@si\endcsname{\let\PY@bf=\textbf\def\PY@tc##1{\textcolor[rgb]{0.73,0.40,0.53}{##1}}}
\expandafter\def\csname PY@tok@s2\endcsname{\def\PY@tc##1{\textcolor[rgb]{0.73,0.13,0.13}{##1}}}
\expandafter\def\csname PY@tok@nt\endcsname{\let\PY@bf=\textbf\def\PY@tc##1{\textcolor[rgb]{0.00,0.50,0.00}{##1}}}
\expandafter\def\csname PY@tok@nv\endcsname{\def\PY@tc##1{\textcolor[rgb]{0.10,0.09,0.49}{##1}}}
\expandafter\def\csname PY@tok@s1\endcsname{\def\PY@tc##1{\textcolor[rgb]{0.73,0.13,0.13}{##1}}}
\expandafter\def\csname PY@tok@dl\endcsname{\def\PY@tc##1{\textcolor[rgb]{0.73,0.13,0.13}{##1}}}
\expandafter\def\csname PY@tok@ch\endcsname{\let\PY@it=\textit\def\PY@tc##1{\textcolor[rgb]{0.25,0.50,0.50}{##1}}}
\expandafter\def\csname PY@tok@m\endcsname{\def\PY@tc##1{\textcolor[rgb]{0.40,0.40,0.40}{##1}}}
\expandafter\def\csname PY@tok@gp\endcsname{\let\PY@bf=\textbf\def\PY@tc##1{\textcolor[rgb]{0.00,0.00,0.50}{##1}}}
\expandafter\def\csname PY@tok@sh\endcsname{\def\PY@tc##1{\textcolor[rgb]{0.73,0.13,0.13}{##1}}}
\expandafter\def\csname PY@tok@ow\endcsname{\let\PY@bf=\textbf\def\PY@tc##1{\textcolor[rgb]{0.67,0.13,1.00}{##1}}}
\expandafter\def\csname PY@tok@sx\endcsname{\def\PY@tc##1{\textcolor[rgb]{0.00,0.50,0.00}{##1}}}
\expandafter\def\csname PY@tok@bp\endcsname{\def\PY@tc##1{\textcolor[rgb]{0.00,0.50,0.00}{##1}}}
\expandafter\def\csname PY@tok@c1\endcsname{\let\PY@it=\textit\def\PY@tc##1{\textcolor[rgb]{0.25,0.50,0.50}{##1}}}
\expandafter\def\csname PY@tok@fm\endcsname{\def\PY@tc##1{\textcolor[rgb]{0.00,0.00,1.00}{##1}}}
\expandafter\def\csname PY@tok@o\endcsname{\def\PY@tc##1{\textcolor[rgb]{0.40,0.40,0.40}{##1}}}
\expandafter\def\csname PY@tok@kc\endcsname{\let\PY@bf=\textbf\def\PY@tc##1{\textcolor[rgb]{0.00,0.50,0.00}{##1}}}
\expandafter\def\csname PY@tok@c\endcsname{\let\PY@it=\textit\def\PY@tc##1{\textcolor[rgb]{0.25,0.50,0.50}{##1}}}
\expandafter\def\csname PY@tok@mf\endcsname{\def\PY@tc##1{\textcolor[rgb]{0.40,0.40,0.40}{##1}}}
\expandafter\def\csname PY@tok@err\endcsname{\def\PY@bc##1{\setlength{\fboxsep}{0pt}\fcolorbox[rgb]{1.00,0.00,0.00}{1,1,1}{\strut ##1}}}
\expandafter\def\csname PY@tok@mb\endcsname{\def\PY@tc##1{\textcolor[rgb]{0.40,0.40,0.40}{##1}}}
\expandafter\def\csname PY@tok@ss\endcsname{\def\PY@tc##1{\textcolor[rgb]{0.10,0.09,0.49}{##1}}}
\expandafter\def\csname PY@tok@sr\endcsname{\def\PY@tc##1{\textcolor[rgb]{0.73,0.40,0.53}{##1}}}
\expandafter\def\csname PY@tok@mo\endcsname{\def\PY@tc##1{\textcolor[rgb]{0.40,0.40,0.40}{##1}}}
\expandafter\def\csname PY@tok@kd\endcsname{\let\PY@bf=\textbf\def\PY@tc##1{\textcolor[rgb]{0.00,0.50,0.00}{##1}}}
\expandafter\def\csname PY@tok@mi\endcsname{\def\PY@tc##1{\textcolor[rgb]{0.40,0.40,0.40}{##1}}}
\expandafter\def\csname PY@tok@kn\endcsname{\let\PY@bf=\textbf\def\PY@tc##1{\textcolor[rgb]{0.00,0.50,0.00}{##1}}}
\expandafter\def\csname PY@tok@cpf\endcsname{\let\PY@it=\textit\def\PY@tc##1{\textcolor[rgb]{0.25,0.50,0.50}{##1}}}
\expandafter\def\csname PY@tok@kr\endcsname{\let\PY@bf=\textbf\def\PY@tc##1{\textcolor[rgb]{0.00,0.50,0.00}{##1}}}
\expandafter\def\csname PY@tok@s\endcsname{\def\PY@tc##1{\textcolor[rgb]{0.73,0.13,0.13}{##1}}}
\expandafter\def\csname PY@tok@kp\endcsname{\def\PY@tc##1{\textcolor[rgb]{0.00,0.50,0.00}{##1}}}
\expandafter\def\csname PY@tok@w\endcsname{\def\PY@tc##1{\textcolor[rgb]{0.73,0.73,0.73}{##1}}}
\expandafter\def\csname PY@tok@kt\endcsname{\def\PY@tc##1{\textcolor[rgb]{0.69,0.00,0.25}{##1}}}
\expandafter\def\csname PY@tok@sc\endcsname{\def\PY@tc##1{\textcolor[rgb]{0.73,0.13,0.13}{##1}}}
\expandafter\def\csname PY@tok@sb\endcsname{\def\PY@tc##1{\textcolor[rgb]{0.73,0.13,0.13}{##1}}}
\expandafter\def\csname PY@tok@sa\endcsname{\def\PY@tc##1{\textcolor[rgb]{0.73,0.13,0.13}{##1}}}
\expandafter\def\csname PY@tok@k\endcsname{\let\PY@bf=\textbf\def\PY@tc##1{\textcolor[rgb]{0.00,0.50,0.00}{##1}}}
\expandafter\def\csname PY@tok@se\endcsname{\let\PY@bf=\textbf\def\PY@tc##1{\textcolor[rgb]{0.73,0.40,0.13}{##1}}}
\expandafter\def\csname PY@tok@sd\endcsname{\let\PY@it=\textit\def\PY@tc##1{\textcolor[rgb]{0.73,0.13,0.13}{##1}}}

\def\PYZbs{\char`\\}
\def\PYZus{\char`\_}
\def\PYZob{\char`\{}
\def\PYZcb{\char`\}}
\def\PYZca{\char`\^}
\def\PYZam{\char`\&}
\def\PYZlt{\char`\<}
\def\PYZgt{\char`\>}
\def\PYZsh{\char`\#}
\def\PYZpc{\char`\%}
\def\PYZdl{\char`\$}
\def\PYZhy{\char`\-}
\def\PYZsq{\char`\'}
\def\PYZdq{\char`\"}
\def\PYZti{\char`\~}
% for compatibility with earlier versions
\def\PYZat{@}
\def\PYZlb{[}
\def\PYZrb{]}
\makeatother


    % For linebreaks inside Verbatim environment from package fancyvrb. 
    \makeatletter
        \newbox\Wrappedcontinuationbox 
        \newbox\Wrappedvisiblespacebox 
        \newcommand*\Wrappedvisiblespace {\textcolor{red}{\textvisiblespace}} 
        \newcommand*\Wrappedcontinuationsymbol {\textcolor{red}{\llap{\tiny$\m@th\hookrightarrow$}}} 
        \newcommand*\Wrappedcontinuationindent {3ex } 
        \newcommand*\Wrappedafterbreak {\kern\Wrappedcontinuationindent\copy\Wrappedcontinuationbox} 
        % Take advantage of the already applied Pygments mark-up to insert 
        % potential linebreaks for TeX processing. 
        %        {, <, #, %, $, ' and ": go to next line. 
        %        _, }, ^, &, >, - and ~: stay at end of broken line. 
        % Use of \textquotesingle for straight quote. 
        \newcommand*\Wrappedbreaksatspecials {% 
            \def\PYGZus{\discretionary{\char`\_}{\Wrappedafterbreak}{\char`\_}}% 
            \def\PYGZob{\discretionary{}{\Wrappedafterbreak\char`\{}{\char`\{}}% 
            \def\PYGZcb{\discretionary{\char`\}}{\Wrappedafterbreak}{\char`\}}}% 
            \def\PYGZca{\discretionary{\char`\^}{\Wrappedafterbreak}{\char`\^}}% 
            \def\PYGZam{\discretionary{\char`\&}{\Wrappedafterbreak}{\char`\&}}% 
            \def\PYGZlt{\discretionary{}{\Wrappedafterbreak\char`\<}{\char`\<}}% 
            \def\PYGZgt{\discretionary{\char`\>}{\Wrappedafterbreak}{\char`\>}}% 
            \def\PYGZsh{\discretionary{}{\Wrappedafterbreak\char`\#}{\char`\#}}% 
            \def\PYGZpc{\discretionary{}{\Wrappedafterbreak\char`\%}{\char`\%}}% 
            \def\PYGZdl{\discretionary{}{\Wrappedafterbreak\char`\$}{\char`\$}}% 
            \def\PYGZhy{\discretionary{\char`\-}{\Wrappedafterbreak}{\char`\-}}% 
            \def\PYGZsq{\discretionary{}{\Wrappedafterbreak\textquotesingle}{\textquotesingle}}% 
            \def\PYGZdq{\discretionary{}{\Wrappedafterbreak\char`\"}{\char`\"}}% 
            \def\PYGZti{\discretionary{\char`\~}{\Wrappedafterbreak}{\char`\~}}% 
        } 
        % Some characters . , ; ? ! / are not pygmentized. 
        % This macro makes them "active" and they will insert potential linebreaks 
        \newcommand*\Wrappedbreaksatpunct {% 
            \lccode`\~`\.\lowercase{\def~}{\discretionary{\hbox{\char`\.}}{\Wrappedafterbreak}{\hbox{\char`\.}}}% 
            \lccode`\~`\,\lowercase{\def~}{\discretionary{\hbox{\char`\,}}{\Wrappedafterbreak}{\hbox{\char`\,}}}% 
            \lccode`\~`\;\lowercase{\def~}{\discretionary{\hbox{\char`\;}}{\Wrappedafterbreak}{\hbox{\char`\;}}}% 
            \lccode`\~`\:\lowercase{\def~}{\discretionary{\hbox{\char`\:}}{\Wrappedafterbreak}{\hbox{\char`\:}}}% 
            \lccode`\~`\?\lowercase{\def~}{\discretionary{\hbox{\char`\?}}{\Wrappedafterbreak}{\hbox{\char`\?}}}% 
            \lccode`\~`\!\lowercase{\def~}{\discretionary{\hbox{\char`\!}}{\Wrappedafterbreak}{\hbox{\char`\!}}}% 
            \lccode`\~`\/\lowercase{\def~}{\discretionary{\hbox{\char`\/}}{\Wrappedafterbreak}{\hbox{\char`\/}}}% 
            \catcode`\.\active
            \catcode`\,\active 
            \catcode`\;\active
            \catcode`\:\active
            \catcode`\?\active
            \catcode`\!\active
            \catcode`\/\active 
            \lccode`\~`\~ 	
        }
    \makeatother

    \let\OriginalVerbatim=\Verbatim
    \makeatletter
    \renewcommand{\Verbatim}[1][1]{%
        %\parskip\z@skip
        \sbox\Wrappedcontinuationbox {\Wrappedcontinuationsymbol}%
        \sbox\Wrappedvisiblespacebox {\FV@SetupFont\Wrappedvisiblespace}%
        \def\FancyVerbFormatLine ##1{\hsize\linewidth
            \vtop{\raggedright\hyphenpenalty\z@\exhyphenpenalty\z@
                \doublehyphendemerits\z@\finalhyphendemerits\z@
                \strut ##1\strut}%
        }%
        % If the linebreak is at a space, the latter will be displayed as visible
        % space at end of first line, and a continuation symbol starts next line.
        % Stretch/shrink are however usually zero for typewriter font.
        \def\FV@Space {%
            \nobreak\hskip\z@ plus\fontdimen3\font minus\fontdimen4\font
            \discretionary{\copy\Wrappedvisiblespacebox}{\Wrappedafterbreak}
            {\kern\fontdimen2\font}%
        }%
        
        % Allow breaks at special characters using \PYG... macros.
        \Wrappedbreaksatspecials
        % Breaks at punctuation characters . , ; ? ! and / need catcode=\active 	
        \OriginalVerbatim[#1,codes*=\Wrappedbreaksatpunct]%
    }
    \makeatother

    % Exact colors from NB
    \definecolor{incolor}{HTML}{303F9F}
    \definecolor{outcolor}{HTML}{D84315}
    \definecolor{cellborder}{HTML}{CFCFCF}
    \definecolor{cellbackground}{HTML}{F7F7F7}
    
    % prompt
    \newcommand{\prompt}[4]{
        \llap{{\color{#2}[#3]: #4}}\vspace{-1.25em}
    }
    

    
    % Prevent overflowing lines due to hard-to-break entities
    \sloppy 
    % Setup hyperref package
    \hypersetup{
      breaklinks=true,  % so long urls are correctly broken across lines
      colorlinks=true,
      urlcolor=urlcolor,
      linkcolor=linkcolor,
      citecolor=citecolor,
      }
    % Slightly bigger margins than the latex defaults
    
    \geometry{verbose,tmargin=1in,bmargin=1in,lmargin=1in,rmargin=1in}
    
    

    \begin{document}
    
    
    \maketitle
    
    

    
    \section{Example 1: Linear
regression}\label{example-1-linear-regression}

\subsubsection{Description of ToothGrowth
dataset}\label{description-of-toothgrowth-dataset}

The ToothGrowth dataset loads with the datasets package.

    The data are from an experiment examining how vitamin C dosage delivered
in 2 different methods predicts tooth growth in guinea pigs.

The data consist of 60 observations, representing 60 guinea pigs, and 3
variables:

\begin{itemize}
\tightlist
\item
  \textbf{\emph{len}}: numeric, tooth (odontoblast, actually) length
\item
  \textbf{\emph{supp}}: factor, supplement type, 2 levels, ``VC'' is
  ascorbic acid, and ``OJ'' is orange juice
\item
  \textbf{\emph{dose}}: numeric, dose (mg/day)
\end{itemize}

A quick look at ToothGrowth reveals that many guinea pigs were given the
same dose of vitamin C. Three doses, 0.5, 1.0, and 2.0, were used. ​

    \begin{tcolorbox}[breakable, size=fbox, boxrule=1pt, pad at break*=1mm,colback=cellbackground, colframe=cellborder]
\prompt{In}{incolor}{ }{\hspace{4pt}}
\begin{Verbatim}[commandchars=\\\{\}]
\PY{c+c1}{\PYZsh{}look at ToothGrowth data}
str\PY{p}{(}ToothGrowth\PY{p}{)}
\end{Verbatim}
\end{tcolorbox}

    \begin{tcolorbox}[breakable, size=fbox, boxrule=1pt, pad at break*=1mm,colback=cellbackground, colframe=cellborder]
\prompt{In}{incolor}{ }{\hspace{4pt}}
\begin{Verbatim}[commandchars=\\\{\}]
\PY{k+kp}{unique}\PY{p}{(}ToothGrowth\PY{o}{\PYZdl{}}dose\PY{p}{)}
\PY{c+c1}{\PYZsh{}\PYZsh{} [1] 0.5 1.0 2.0}
\end{Verbatim}
\end{tcolorbox}

    \subsubsection{Example 1: Exploratory
analysis}\label{example-1-exploratory-analysis}

Graphical exploration of the dataset provides the researcher with
descriptive depictions of summaries of variables or relationship among
variables.

We being our exploration of Toothgrowth by examining the distributions
of supp and dose. Since both variables are discretely distributed, we
want the frequencies (counts) of each value.

Frequencies are often plotted as bars, so we select geom\_bar. We
specify dose mapped to x and supp mapped to the fill color of the bars,
which will give us counts of each dose divided by supp.

\subsubsection{Exploring distributions}\label{exploring-distributions}

    \begin{tcolorbox}[breakable, size=fbox, boxrule=1pt, pad at break*=1mm,colback=cellbackground, colframe=cellborder]
\prompt{In}{incolor}{3}{\hspace{4pt}}
\begin{Verbatim}[commandchars=\\\{\}]
\PY{k+kn}{library}\PY{p}{(}ggplot2\PY{p}{)}
\PY{c+c1}{\PYZsh{}bar plot of counts of dose by supp}
\PY{c+c1}{\PYZsh{}data are balanced, so not so interesting}
ggplot\PY{p}{(}ToothGrowth\PY{p}{,} aes\PY{p}{(}x\PY{o}{=}dose\PY{p}{,} fill\PY{o}{=}supp\PY{p}{)}\PY{p}{)} \PY{o}{+}  geom\PYZus{}bar\PY{p}{(}\PY{p}{)}
\end{Verbatim}
\end{tcolorbox}

    
    
    \begin{center}
    \adjustimage{max size={0.9\linewidth}{0.9\paperheight}}{output_5_1.png}
    \end{center}
    { \hspace*{\fill} \\}
    
    Statistical models often make assumptions about the distribution of the
outcome (or its residuals), so an inspection might be wise. First let's
check the overall distribution of ``len'' with a density plot:

    \begin{tcolorbox}[breakable, size=fbox, boxrule=1pt, pad at break*=1mm,colback=cellbackground, colframe=cellborder]
\prompt{In}{incolor}{7}{\hspace{4pt}}
\begin{Verbatim}[commandchars=\\\{\}]
\PY{c+c1}{\PYZsh{}density of len}
ggplot\PY{p}{(}ToothGrowth\PY{p}{,} aes\PY{p}{(}x\PY{o}{=}len\PY{p}{)}\PY{p}{)}  \PY{o}{+} geom\PYZus{}density\PY{p}{(}\PY{p}{)}
\end{Verbatim}
\end{tcolorbox}

    
    
    \begin{center}
    \adjustimage{max size={0.9\linewidth}{0.9\paperheight}}{output_7_1.png}
    \end{center}
    { \hspace*{\fill} \\}
    
    We can get densities of len by supp by mapping supp to color:

    \begin{tcolorbox}[breakable, size=fbox, boxrule=1pt, pad at break*=1mm,colback=cellbackground, colframe=cellborder]
\prompt{In}{incolor}{4}{\hspace{4pt}}
\begin{Verbatim}[commandchars=\\\{\}]
\PY{k+kn}{library}\PY{p}{(}ggplot2\PY{p}{)}
\PY{c+c1}{\PYZsh{}density of len by supp}
ggplot\PY{p}{(}ToothGrowth\PY{p}{,} aes\PY{p}{(}x\PY{o}{=}len\PY{p}{,} color\PY{o}{=}supp\PY{p}{)}\PY{p}{)} \PY{o}{+} 
  geom\PYZus{}density\PY{p}{(}\PY{p}{)}
\end{Verbatim}
\end{tcolorbox}

    
    
    \begin{center}
    \adjustimage{max size={0.9\linewidth}{0.9\paperheight}}{output_9_1.png}
    \end{center}
    { \hspace*{\fill} \\}
    
    \paragraph{Summarizing the outcome across a
predictor}\label{summarizing-the-outcome-across-a-predictor}

Because dose takes on only 3 values, many point are crowded in 3
columns, obscuring the shape of relationship between dose and len. We
replace the column of points at each dose value with its mean and
confidence limit using stat\_summary instead of geom\_point.

    The outcome distributions appear a bit skewed, but the samples are
small.

\subsubsection{Quick scatterplot of outcome and
predictor}\label{quick-scatterplot-of-outcome-and-predictor}

We plot the dose-tooth length (len) relationship.

    \begin{tcolorbox}[breakable, size=fbox, boxrule=1pt, pad at break*=1mm,colback=cellbackground, colframe=cellborder]
\prompt{In}{incolor}{5}{\hspace{4pt}}
\begin{Verbatim}[commandchars=\\\{\}]
\PY{c+c1}{\PYZsh{}not the best scatterplot}
tp \PY{o}{\PYZlt{}\PYZhy{}} ggplot\PY{p}{(}ToothGrowth\PY{p}{,} aes\PY{p}{(}x\PY{o}{=}dose\PY{p}{,} y\PY{o}{=}len\PY{p}{)}\PY{p}{)}
tp \PY{o}{+} geom\PYZus{}point\PY{p}{(}\PY{p}{)}
\end{Verbatim}
\end{tcolorbox}

    
    
    \begin{center}
    \adjustimage{max size={0.9\linewidth}{0.9\paperheight}}{output_12_1.png}
    \end{center}
    { \hspace*{\fill} \\}
    
    \begin{tcolorbox}[breakable, size=fbox, boxrule=1pt, pad at break*=1mm,colback=cellbackground, colframe=cellborder]
\prompt{In}{incolor}{6}{\hspace{4pt}}
\begin{Verbatim}[commandchars=\\\{\}]
\PY{c+c1}{\PYZsh{}mean and cl of len at each dose}
tp.1 \PY{o}{\PYZlt{}\PYZhy{}} tp \PY{o}{+} stat\PYZus{}summary\PY{p}{(}fun.data\PY{o}{=}\PY{l+s}{\PYZdq{}}\PY{l+s}{mean\PYZus{}cl\PYZus{}normal\PYZdq{}}\PY{p}{)}
tp.1
\end{Verbatim}
\end{tcolorbox}

    
    
    \begin{center}
    \adjustimage{max size={0.9\linewidth}{0.9\paperheight}}{output_13_1.png}
    \end{center}
    { \hspace*{\fill} \\}
    
    An additional call to stat\_summary with fun.y=mean (fun.y because mean
returns one value) and changing the geom to ``line'' adds a line between
means. ​

    \begin{tcolorbox}[breakable, size=fbox, boxrule=1pt, pad at break*=1mm,colback=cellbackground, colframe=cellborder]
\prompt{In}{incolor}{7}{\hspace{4pt}}
\begin{Verbatim}[commandchars=\\\{\}]
\PY{c+c1}{\PYZsh{}add a line plot of means to see dose\PYZhy{}len relationship}
\PY{c+c1}{\PYZsh{}maybe not linear}
tp.2 \PY{o}{\PYZlt{}\PYZhy{}} tp.1 \PY{o}{+} stat\PYZus{}summary\PY{p}{(}fun.y\PY{o}{=}\PY{l+s}{\PYZdq{}}\PY{l+s}{mean\PYZdq{}}\PY{p}{,} geom\PY{o}{=}\PY{l+s}{\PYZdq{}}\PY{l+s}{line\PYZdq{}}\PY{p}{)}
tp.2
\end{Verbatim}
\end{tcolorbox}

    
    
    \begin{center}
    \adjustimage{max size={0.9\linewidth}{0.9\paperheight}}{output_15_1.png}
    \end{center}
    { \hspace*{\fill} \\}
    
    \begin{tcolorbox}[breakable, size=fbox, boxrule=1pt, pad at break*=1mm,colback=cellbackground, colframe=cellborder]
\prompt{In}{incolor}{ }{\hspace{4pt}}
\begin{Verbatim}[commandchars=\\\{\}]
Does a third variable moderate the x\PY{o}{\PYZhy{}}y relationship\PY{o}{?}

Does the dose\PY{o}{\PYZhy{}}len relationship depend on supp\PY{o}{?} We can specify new global aesthetics \PY{k+kr}{in} aes.
\end{Verbatim}
\end{tcolorbox}

    \begin{tcolorbox}[breakable, size=fbox, boxrule=1pt, pad at break*=1mm,colback=cellbackground, colframe=cellborder]
\prompt{In}{incolor}{ }{\hspace{4pt}}
\begin{Verbatim}[commandchars=\\\{\}]
\PY{c+c1}{\PYZsh{}all plots in tp.2 will now be colored by supp}
tp.2 \PY{o}{+} aes\PY{p}{(}color\PY{o}{=}supp\PY{p}{)}
\end{Verbatim}
\end{tcolorbox}

    \subsubsection{Interpreting the previous
graph}\label{interpreting-the-previous-graph}

This graph suggests:

\begin{enumerate}
\def\labelenumi{\arabic{enumi}.}
\tightlist
\item
  The slope of the dose-response curve decreases as dose increases for
  both supp types, suggesting a quadratic function.
\item
  The slope of the OJ curve flattens more dramatically, perhaps
  suggesting the quadratic term is different between supplement groups
\item
  The initial slopes look rather similar, perhaps suggesting that the
  linear term may not be different between groups
\item
  The 2 supp group means differ at the two lower doses, but not at the
  highest dose
\end{enumerate}

ggplot2 makes graphs summarizing the outcome easy ​ We just plotted
means and confidence limits of len, with lines connecting the means,
separated by supp, all without any manipulation to the original data! ​
The stat\_summary function facilitates looking at patterns of means,
just as regression models do. ​ Next we fit our linear regression model
and check model assumptions with diagnostic graphs. ​

    \subsubsection{Model preliminaries}\label{model-preliminaries}

​ We want to model how tooth length (len) is predicted by dose, allowing
for moderation of this relationship through an interaction with supp. ​
We assume that dose and tooth length have a smooth, continuous
relationship in the range of doses tested, so we will treat dose as a
continuous (numeric) predictor. We also create a dose-squared variable,
for use in the quadratic model and prediction later. ​

    \begin{tcolorbox}[breakable, size=fbox, boxrule=1pt, pad at break*=1mm,colback=cellbackground, colframe=cellborder]
\prompt{In}{incolor}{8}{\hspace{4pt}}
\begin{Verbatim}[commandchars=\\\{\}]
\PY{c+c1}{\PYZsh{}create dose\PYZhy{}squared variable}
ToothGrowth\PY{o}{\PYZdl{}}dosesq \PY{o}{\PYZlt{}\PYZhy{}} ToothGrowth\PY{o}{\PYZdl{}}dose\PY{o}{\PYZca{}}\PY{l+m}{2}
\end{Verbatim}
\end{tcolorbox}

    \subsubsection{Fitting the model}\label{fitting-the-model}

We noticed in the previous graph that the dose-len relationship appears
quadratic, that the quadratic effect may differ between supp groups, but
that the initial linear slope may not be so different. We fit a model to
reflect our expectations:

    \begin{tcolorbox}[breakable, size=fbox, boxrule=1pt, pad at break*=1mm,colback=cellbackground, colframe=cellborder]
\prompt{In}{incolor}{9}{\hspace{4pt}}
\begin{Verbatim}[commandchars=\\\{\}]
lm2 \PY{o}{\PYZlt{}\PYZhy{}} lm\PY{p}{(}len \PY{o}{\PYZti{}} dose \PY{o}{+} dosesq\PY{o}{*}supp\PY{p}{,} data\PY{o}{=}ToothGrowth\PY{p}{)}
\PY{k+kp}{summary}\PY{p}{(}lm2\PY{p}{)}\PY{o}{\PYZdl{}}coef
\PY{c+c1}{\PYZsh{}\PYZsh{}                 Estimate Std. Error    t value     Pr(\PYZgt{}|t|)}
\PY{c+c1}{\PYZsh{}\PYZsh{} (Intercept)    0.7491667  2.7983895  0.2677135 7.899213e\PYZhy{}01}
\PY{c+c1}{\PYZsh{}\PYZsh{} dose          30.1550000  5.2474684  5.7465806 4.114588e\PYZhy{}07}
\PY{c+c1}{\PYZsh{}\PYZsh{} dosesq        \PYZhy{}8.7238095  2.0402571 \PYZhy{}4.2758383 7.640686e\PYZhy{}05}
\PY{c+c1}{\PYZsh{}\PYZsh{} suppVC        \PYZhy{}6.4783333  1.3762287 \PYZhy{}4.7073088 1.739152e\PYZhy{}05}
\PY{c+c1}{\PYZsh{}\PYZsh{} dosesq:suppVC  1.5876190  0.5770719  2.7511635 8.024694e\PYZhy{}03}
\end{Verbatim}
\end{tcolorbox}

    \begin{tabular}{r|llll}
  & Estimate & Std. Error & t value & Pr(>\textbar{}t\textbar{})\\
\hline
	(Intercept) &  0.7491667   & 2.7983895    &  0.2677135   & 7.899213e-01\\
	dose & 30.1550000   & 5.2474684    &  5.7465806   & 4.114588e-07\\
	dosesq & -8.7238095   & 2.0402571    & -4.2758383   & 7.640686e-05\\
	suppVC & -6.4783333   & 1.3762287    & -4.7073088   & 1.739152e-05\\
	dosesq:suppVC &  1.5876190   & 0.5770719    &  2.7511635   & 8.024694e-03\\
\end{tabular}


    
    The model appears to conform to our expectations.

    \subsubsection{Example 1: Model
diagnostics}\label{example-1-model-diagnostics}

Statistical inference depends on the assumptions of the regression
model, which we check with diagnostic graphs. Common diagnostics for
linear regression

\begin{itemize}
\tightlist
\item
  inspect the normality of the residuals
\item
  verify that the residuals show no trends (assumption of linearity) and
  are homoscedastic,
\item
  check for overly influential outliers.
\item
  fortify makes linear regression diagnostics easy
\end{itemize}

    Conveniently, the fortify function takes a lm model object (one among
several classes) and returns a dataset with several estimated diagnostic
variables including:

\begin{itemize}
\tightlist
\item
  .hat: leverages(influence)
\item
  .sigma: residual standard deviation when observation dropped from
  model
\item
  .cooksd: Cook's distance
\item
  .fitted: fitted (predicted) values
\item
  .resid: residuals
\item
  .stdresid: standardized residuals
\end{itemize}

We fortify our lm2 object with these diagnostic variables and take a
quick look a the new variables.

    \begin{tcolorbox}[breakable, size=fbox, boxrule=1pt, pad at break*=1mm,colback=cellbackground, colframe=cellborder]
\prompt{In}{incolor}{11}{\hspace{4pt}}
\begin{Verbatim}[commandchars=\\\{\}]
\PY{c+c1}{\PYZsh{}create dataset with original data and diagnostic variables}
lm2diag \PY{o}{\PYZlt{}\PYZhy{}} fortify\PY{p}{(}lm2\PY{p}{)}
\PY{c+c1}{\PYZsh{}quick look}
\PY{k+kp}{head}\PY{p}{(}lm2diag\PY{p}{)}
\end{Verbatim}
\end{tcolorbox}

    \begin{tabular}{r|llllllllll}
 len & dose & dosesq & supp & .hat & .sigma & .cooksd & .fitted & .resid & .stdresid\\
\hline
	  4.2         & 0.5          & 0.25         & VC           & 0.08095238   & 3.623134     & 0.0165458459 & 7.564286     & -3.3642857   & -0.96913367 \\
	 11.5         & 0.5          & 0.25         & VC           & 0.08095238   & 3.611516     & 0.0226438547 & 7.564286     &  3.9357143   &  1.13374237 \\
	  7.3         & 0.5          & 0.25         & VC           & 0.08095238   & 3.654279     & 0.0001021058 & 7.564286     & -0.2642857   & -0.07613152 \\
	  5.8         & 0.5          & 0.25         & VC           & 0.08095238   & 3.645881     & 0.0045503109 & 7.564286     & -1.7642857   & -0.50822934 \\
	  6.4         & 0.5          & 0.25         & VC           & 0.08095238   & 3.650733     & 0.0019816291 & 7.564286     & -1.1642857   & -0.33539021 \\
	 10.0         & 0.5          & 0.25         & VC           & 0.08095238   & 3.638080     & 0.0086727319 & 7.564286     &  2.4357143   &  0.70164455 \\
\end{tabular}


    
    \subsubsection{Normality of residuals: q-q plot and
stat\_qq}\label{normality-of-residuals-q-q-plot-and-statux5fqq}

\begin{itemize}
\item
  A q-q plot can assess the assumption that the residuals are normally
  distributed by plotting the standardized residuals (observed z-score)
  against theoretical quantiles of the normal distribution (expected
  z-score if normally distributed).
\item
  stat\_qq creates a qq-plot. The only required aesthetic is sample,
  which we map to the standardized residual variable created by fortify,
  .stdresid.
\item
  A diagonal reference line (intercept=0, slope=1) is added to the plot
  with geom\_abline, representing perfect fit to a normal distribution.
  The normal distribution will appear to be a reasonable fit below.
\end{itemize}

    \begin{tcolorbox}[breakable, size=fbox, boxrule=1pt, pad at break*=1mm,colback=cellbackground, colframe=cellborder]
\prompt{In}{incolor}{12}{\hspace{4pt}}
\begin{Verbatim}[commandchars=\\\{\}]
\PY{c+c1}{\PYZsh{}q\PYZhy{}q plot of residuals and diagonal reference line}
\PY{c+c1}{\PYZsh{}geom\PYZus{}abline default aesthetics are yintercept=0 and slope=1}
ggplot\PY{p}{(}lm2diag\PY{p}{,} aes\PY{p}{(}sample\PY{o}{=}\PY{l+m}{.}stdresid\PY{p}{)}\PY{p}{)} \PY{o}{+} 
  stat\PYZus{}qq\PY{p}{(}\PY{p}{)} \PY{o}{+} 
  geom\PYZus{}abline\PY{p}{(}\PY{p}{)}
\end{Verbatim}
\end{tcolorbox}

    
    
    \begin{center}
    \adjustimage{max size={0.9\linewidth}{0.9\paperheight}}{output_28_1.png}
    \end{center}
    { \hspace*{\fill} \\}
    
    \begin{tcolorbox}[breakable, size=fbox, boxrule=1pt, pad at break*=1mm,colback=cellbackground, colframe=cellborder]
\prompt{In}{incolor}{ }{\hspace{4pt}}
\begin{Verbatim}[commandchars=\\\{\}]
\PY{c+c1}{\PYZsh{}\PYZsh{}\PYZsh{} Linearity and Homoscedasticity: residuals vs fitted}

We \PY{k+kr}{next} assess the assumptions of homoscedescasticity and linear relationships between the outcome and predictors. A residuals vs fitted \PY{p}{(}predicted value\PY{p}{)} plot assesses both of these assmuptions.

An even spread of residuals around \PY{l+m}{0} suggests homoscedasticity\PY{p}{,} and a zero\PY{p}{,} flat slope \PY{k+kr}{for} residuals against fitted suggests linearity of predictor effects.
\end{Verbatim}
\end{tcolorbox}

    \begin{tcolorbox}[breakable, size=fbox, boxrule=1pt, pad at break*=1mm,colback=cellbackground, colframe=cellborder]
\prompt{In}{incolor}{ }{\hspace{4pt}}
\begin{Verbatim}[commandchars=\\\{\}]
We build our residuals vs fitted plot like so\PY{o}{:}

\PY{l+m}{1}\PY{l+m}{.} Start with a scatter plot of x\PY{o}{=}\PY{l+m}{.}fitted and y\PY{o}{=}\PY{l+m}{.}stdresid.
\PY{l+m}{2}\PY{l+m}{.} Add a plot the means and standard errors of the residuals across fitted values using stat\PYZus{}summary. The standard error bars somewhat address homoskedasticity.
\PY{l+m}{3}\PY{l+m}{.} Plot a line representing the mean trend of the residuals with another call to stat\PYZus{}summary \PY{p}{(}changing \PY{k+kr}{function} to mean and geom to line\PY{p}{)}\PY{l+m}{.} Normally\PY{p}{,} we would use geom\PYZus{}smooth to plot a loess curve to visualize the trend among residuals\PY{p}{,} but loess smooths do not work well when the variable mapped to x is discrete.
\PY{l+m}{4}\PY{l+m}{.} The error bars do not appear to vary too much and the line seems sufficiently flat to feel that neither assumption has been violated.
\end{Verbatim}
\end{tcolorbox}

    \begin{tcolorbox}[breakable, size=fbox, boxrule=1pt, pad at break*=1mm,colback=cellbackground, colframe=cellborder]
\prompt{In}{incolor}{ }{\hspace{4pt}}
\begin{Verbatim}[commandchars=\\\{\}]
\PY{c+c1}{\PYZsh{}residual vs fitted, means and s.e.}
ggplot\PY{p}{(}lm2diag\PY{p}{,} aes\PY{p}{(}x\PY{o}{=}\PY{l+m}{.}fitted\PY{p}{,} y\PY{o}{=}\PY{l+m}{.}stdresid\PY{p}{)}\PY{p}{)} \PY{o}{+} 
  geom\PYZus{}point\PY{p}{(}\PY{p}{)} \PY{o}{+} 
  stat\PYZus{}summary\PY{p}{(}\PY{p}{)} \PY{o}{+} 
  stat\PYZus{}summary\PY{p}{(}fun.y\PY{o}{=}\PY{l+s}{\PYZdq{}}\PY{l+s}{mean\PYZdq{}}\PY{p}{,} geom\PY{o}{=}\PY{l+s}{\PYZdq{}}\PY{l+s}{line\PYZdq{}}\PY{p}{)}
\PY{c+c1}{\PYZsh{}\PYZsh{} No summary function supplied, defaulting to `mean\PYZus{}se()}
\end{Verbatim}
\end{tcolorbox}

    \subsubsection{Identifying influential
observations}\label{identifying-influential-observations}

Strongly influential observations can distort regression coefficients.
The most influential datapoints will typically have more extreme
predictor values (leverage), measured by .hat, and large residuals. The
overall influence of an observation on the model is measured by Cook's
D, variable .cooksd.

The Toothgrowth dataset is fairly balanced across doses and supps (20 at
each of 3 doses, 30 at each of 2 supp). Thus, no values are ``extreme'',
so influential observations for this model will be those with large
residuals.

In the following plot: * map .hat to x, .stdresid to y * map .cooksd to
size, making more influential points larger. * label the points in
geom\_text with their row numbers for identification

The dependence of Cook's D on both leverage and residual is apparent in
the plot, with Cook's D rising as we move away from center (larger
residual) and to the right (larger leverage). No point looks too
influential for concern.

    \begin{tcolorbox}[breakable, size=fbox, boxrule=1pt, pad at break*=1mm,colback=cellbackground, colframe=cellborder]
\prompt{In}{incolor}{13}{\hspace{4pt}}
\begin{Verbatim}[commandchars=\\\{\}]
\PY{c+c1}{\PYZsh{} in geom\PYZus{}text we SET size to 4 so that size of text is constant,}
\PY{c+c1}{\PYZsh{}   and not tied to .cooksd.  We also  nudge the text .001 (x\PYZhy{}axis units)}
\PY{c+c1}{\PYZsh{}   to the left, and separate overlapping labels}
ggplot\PY{p}{(}lm2diag\PY{p}{,} aes\PY{p}{(}x\PY{o}{=}\PY{l+m}{.}hat\PY{p}{,} y\PY{o}{=}\PY{l+m}{.}stdresid\PY{p}{,} size\PY{o}{=}\PY{l+m}{.}cooksd\PY{p}{)}\PY{p}{)} \PY{o}{+}     
  geom\PYZus{}point\PY{p}{(}\PY{p}{)} \PY{o}{+}
  geom\PYZus{}text\PY{p}{(}aes\PY{p}{(}label\PY{o}{=}\PY{k+kp}{row.names}\PY{p}{(}lm2diag\PY{p}{)}\PY{p}{)}\PY{p}{,} 
             size\PY{o}{=}\PY{l+m}{4}\PY{p}{,} nudge\PYZus{}x\PY{o}{=}\PY{l+m}{\PYZhy{}0.003}\PY{p}{,} check\PYZus{}overlap\PY{o}{=}\PY{n+nb+bp}{T}\PY{p}{)}
\end{Verbatim}
\end{tcolorbox}

    
    
    \begin{center}
    \adjustimage{max size={0.9\linewidth}{0.9\paperheight}}{output_33_1.png}
    \end{center}
    { \hspace*{\fill} \\}
    
    \begin{tcolorbox}[breakable, size=fbox, boxrule=1pt, pad at break*=1mm,colback=cellbackground, colframe=cellborder]
\prompt{In}{incolor}{ }{\hspace{4pt}}
\begin{Verbatim}[commandchars=\\\{\}]

\end{Verbatim}
\end{tcolorbox}


    % Add a bibliography block to the postdoc
    
    
    
    \end{document}
